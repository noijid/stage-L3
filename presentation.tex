\documentclass{beamer}

\usepackage[utf8]{inputenc}
\usepackage{amsmath}
\usepackage{amssymb}
\usepackage{mathrsfs}
\usepackage{graphicx}
\usepackage{layout}
\usepackage{dsfont}
\usepackage[square,numbers,sort&compress]{natbib}
\usepackage[francais]{babel}
%\usepackage[top=2cm, bottom=3cm, left=2cm, right=2cm]{geometry}
\usepackage{listings}
\usepackage{algorithm}
\usepackage{algorithmic}
\usepackage{algorithmique}


\title{Soutenance stage L3 : Factorisation et calcul de logarithmes discrets : les algorithmes de crible}
\author{Oijid Nacim}
\institute{}
\usetheme{Warsaw}
\date{Septembre 2019}

%Macros
\newtheorem{defdef}{Définition}
\newtheorem{nota}{Notation}
\newcommand{\p}{\mathbb{P}} 
\newcommand{\z}{\mathbb{Z}} 
\newcommand{\ztz}{$\mathbb{Z}/2\mathbb{Z}$} 
\newcommand{\al}{\alpha} 
\newcommand{\ere}{\textsuperscript{ère} }
\newcommand{\er}{\textsuperscript{er} }
\newcommand{\eme}{\textsuperscript{ème} }
\newcommand{\HRule}{\rule{\linewidth}{0.5mm}}


\begin{document}

\begin{titlepage}
\vfill
 \begin{center}
  \includegraphics[width = 20mm]{ENS_Lyon.png} \hfill
  \includegraphics[width = 20mm]{cnrs.jpg} \hfill
  \includegraphics[width = 20mm]{LIP.png}\hfill
  \includegraphics[width = 20mm]{univ_lyon.jpg} \hfill
  \includegraphics[width = 20mm]{ucbl.jpg}

\end{center}
\end{titlepage}

\begin{frame}{Introduction}
  

  \begin{itemize}
  
\item<1,2,3,4> La cryptographie

 \item<2,3,4> RSA
  
  \item< 3,4> la place de la factorisation

 \item<4> Les algorithmes naïfs

  \end{itemize}
  
\end{frame}

\begin{frame}{Le crible quadratique}
  
\section{Présentation}
  
\begin{itemize}

 
 \item<1,2,3> Idée : trouver $x$, $y$ tel que $x^2 \equiv y^2 [N]$, on aura alors $kN = (x-y)(x+y)$
 
 \item<2, 3> recherche exaustive des éventuels petits facteurs

 \item<3> cribler pour trouver les grands
 
 
  \end{itemize}
  
\end{frame}

\end{document}



